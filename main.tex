\documentclass{ctexbeamer}
\usetheme{Berlin}

\usepackage{scrextend}

\title{GPU 驱动的发行版适配}
\author{Icenowy Zheng}
\institute{PLCT 实验室}
\date{2023-03-17}

\begin{document}

\frame{\titlepage}

\part{从底向上看 GPU 驱动}
\frame{\partpage}

\begin{frame}
    \frametitle{内核驱动}
    \begin{block}{众所周知}
        只有内核才有直接访问硬件的权限——所谓“用户空间驱动”的存在永远基于内核提供的框架。
    \end{block}
    \begin{itemize}
        \item 显示管线的配置
        \item 显存管理
        \item GPU 任务的提交
        \item 提供额外特性的接口(如硬件编解码等)
    \end{itemize}
\end{frame}

\begin{frame}
    \frametitle{API 实现库}
    \begin{labeling}{通用计算}
        \item [3D 渲染] OpenGL ES/OpenGL/Vulkan
        \item [2D 渲染] OpenVG (及 3D 渲染库)
        \item [资源管理] GBM/EGL
        \item [硬件解码] VA-API/VDPAU/NV\{ENC,DEC\}
        \item [通用计算] OpenCL/Vulkan/CUDA
    \end{labeling}
\end{frame}

\begin{frame}
    \frametitle{(可选)DDX (Device-Dependent X)}
    \begin{alertblock}{注意}
        DDX 仅在 XFree86 X 服务器下适用。
    \end{alertblock}
    显示相关的 DDX 实现了 X11 中,与硬件相关的部分加速特性。
    \begin{block}{DDX 实现的特性举例}
        \begin{itemize}
            \item DRI 认证/缓冲分配(对于 X 客户端调用图形加速所必须)
            \item Xv 叠加层、XRender 2D 加速等特异性加速
        \end{itemize}
    \end{block}
\end{frame}

\begin{frame}
    \frametitle{(可选)实用工具}
    八仙过海,各显神通。
    \begin{block}{举例}
        \begin{labeling}{状态监测类}
            \item [状态监测类] rocm-smi, nvidia-smi
            \item [性能分析类] rocprof, nvprof
        \end{labeling}
    \end{block}
\end{frame}

\part{开源 GPU 驱动的组成}
\frame{\partpage}

\section{内核接口}

\begin{frame}
    \frametitle{内核接口:FBDEV}
    \begin{alertblock}{注意:已废弃}
        内核 FBDEV 子系统已废弃,将不会接受新设备驱动支持;DRM 设备仍可提供 FBDEV 兼容设备节点。
    \end{alertblock}
    真·亮机接口。\\
    仅支持 2D 显示一个简单的 Framebuffer ,没有任何现代化的加速功能。\\
    现在仍旧提供的一大原因是 FBCON 。
\end{frame}

\begin{frame}
    \frametitle{内核接口:DRM (Direct Rendering Manager)}
    最早为了 3D 渲染加速创造的接口;\\
    后来随着 KMS 的引入,也开始能控制显示;\\
    目前开源显示和 GPU 驱动的金标准。\\
    \begin{block}{不同驱动间的接口差异}
        DRM 驱动的 KMS 部分基本是标准化的(虽然允许扩展);但是显存管理、任务管理等均为驱动私有 ioctl 接口,由各个驱动自行定义。
    \end{block}
\end{frame}

\section{API 实现库}

\begin{frame}
    \frametitle{API 实现库:Mesa}
    以一个软件的类 OpenGL\textregistered 实现开局;\\
    后来获得了 DRI 的加成,能够使用图形硬件;\\
    再后来有了 Gallium ,能够让不同驱动之间共享大量基础设施;\\
    时至今日,支持的 API 不断扩张,支持了:\\
    \begin{itemize}
        \item GBM/GLX/EGL
        \item OpenGL (ES 和桌面)/Vulkan
        \item OpenCL
        \item VA-API/VDPAU
        \item 以及其他……
    \end{itemize}
\end{frame}

\section{DDX}

\begin{frame}
    \frametitle{DDX:xf86-video-modesetting}
    “那个”通用的 DDX 。\\
    现已成为 X.Org Server 代码库的一部分。\\
    基于通用的 API 实现:\\
    \begin{itemize}
        \item DRM KMS 内核接口:实现显示管理、Vsync
        \item GBM:实现显存管理
        \item OpenGL (ES):实现 2D 加速(含 2D 渲染加速和视频叠加)
    \end{itemize}
    其中基于 OpenGL (ES) 实现 X11 2D 加速的模块被称为 Glamor 。
\end{frame}

\begin{frame}
    \frametitle{DDX:其他}
    其他许多开源驱动同样也都有对应的 DDX:
    \begin{itemize}
        \item xf86-video-intel
        \item xf86-video-ati
        \item xf86-video-amdgpu
        \item xf86-video-nouveau
    \end{itemize}
    特异 DDX 的主要意义是 2D 加速。
\end{frame}

\part{厂商 GPU 驱动适配问题}
\frame{\partpage}

\section{内核驱动}

\begin{frame}
    \frametitle{闭源内核驱动:也是一件精细活}
    \begin{block}{众所周知}
        Linux 内核不维护内核内部 API 的稳定性。\\
        内核会检查 .ko 文件中的版本信息,并会拒绝不匹配的 .ko 文件。
    \end{block}
    发行闭源驱动的方式:
    \begin{itemize}
        \item 直接丢 .ko 出去——被三体人锁死对应的内核二进制,需要和发行版发布内核的过程同步
        \item 将重要文件编译成 .o 发布——由于内核 API 不稳定,这些文件中对内核 API 的调用依然不保证能跨内核版本可用
        \item 同上,但精心设计 .o 的导入/导出接口——这种情况能够构造出可以跨内核版本的闭源驱动
    \end{itemize}
\end{frame}

\begin{frame}
    \frametitle{闭源内核驱动:许可证问题}
    \begin{block}{众所周知}
        Linux 内核使用 GPLv2 授权,有授权传递性问题。
    \end{block}
    虽然闭源驱动并未被 Linux 赶尽杀绝,但它们的生存环境依然十分艰难……\\
    \begin{block}{EXPORT\_SYMBOL\_GPL}
        Linux 内核将大量接口以 EXPORT\_SYMBOL\_GPL 这个宏导出,而这样的接口将只能被元数据中声明 GPL 兼容许可证的模块调用。
    \end{block}
\end{frame}

\begin{frame}
    \frametitle{闭源内核驱动:接口限制来源}
    Linux 内核中的 GPL 符号限制源于 GPL 中 \textit{derivative work} 的概念;调用限制 GPL 符号的模块被认为需要被传递 GPL ,所以禁止非 GPL 兼容许可证的模块调用。\\
    一般来说, GPL 符号限制应用于 Linux 中一些较有特色的子系统(虽然此方面并无明确标准)。
\end{frame}

\begin{frame}
    \frametitle{闭源内核驱动:观赏厂商的表演}
    为了绕过 Linux 内核所做的限制,各种闭源 GPU 驱动厂商采用了例如下面这些计策:\\
    \begin{labeling}{和平共处}
        \item [和平共处] 基于内核允许闭源模块调用的接口苦苦编写驱动
        \item [掩耳盗铃] 在内核模块元数据中声明 GPL ,即便给出的驱动包含闭源部分
        \item [画地为牢] 编写额外的声明为 GPL 的内核模块作为闭源代码的包装
    \end{labeling}
\end{frame}

\begin{frame}
    \frametitle{闭源内核驱动:绕过手段真的合理吗?}
    \begin{block}{掩耳盗铃}
        当分发这一“声称为 GPL”的内核模块时,如何满足 GPL 对源码的要求呢?
    \end{block}
    \begin{block}{画地为牢}
        此方法已失效:内核版本 5.9 起,Linux 内核将引用非 GPL 兼容模块的模块亦作为非 GPL 兼容模块看待,从而无法使用仅 GPL 符号。\\
        Greg Kroah-Hartman 锐评:
        \begin{quote}
            Ah, the proven-to-be-illegal "GPL Condom" defense :)
        \end{quote}
    \end{block}
\end{frame}

\begin{frame}
    \frametitle{内核驱动:那为什么要闭源呢?}
    \begin{block}{嵌入式 GPU}
        嵌入式 GPU 的通常做法是开源(并在芯片 BSP 中附带)的内核驱动配合闭源的用户空间驱动。\\
        ARM Mali GPU 同时在他们的开发者网站上公开提供未适配特定 SoC 的内核驱动以作为参考。
    \end{block}
    \begin{block}{NVIDIA 的 open-gpu-kernel-modules}
        自 Turing (GeForce 系列对应 RTX20/GTX16)起,NVIDIA 在 GPU 中嵌入了 GSP 处理器,从而能够将需要保密的逻辑移动到 GSP 固件中,而内核驱动本身则不必闭源。
    \end{block}
\end{frame}

\begin{frame}
    \frametitle{内核驱动:警惕无序开发 /dev 的行为}
    另外值得注意的是,很多厂商驱动会在标准的 /dev/fbX 和 /dev/dri/ 之外,创建额外的私有设备节点。\\
    请注意这些私有设备节点的权限,未正确配置可能导致仅 root 用户可以利用 GPU。\\
    编写额外 Udev 规则可以配置这些设备节点的权限,一般来说归属于 root:video 且权限 660 。
\end{frame}

\section{API 实现库}

\begin{frame}
    \frametitle{API 实现库:李代桃僵}
    部分厂商驱动直接提供了替换的 lib\{GL,EGL,GLESv2,gbm\}.so 库文件;例如, NVIDIA 的 340.xx 旧卡驱动。\\
    这种替换系统库文件的厂商驱动危害如下:\\
    \begin{labeling}{污染二进制文件}
        \item [污染包管理器] 若不通过包管理器安装,则有驱动被覆盖之虞;若通过包管理安装,则需要打包时引入替换机制。
        \item [污染二进制文件] 链接可执行文件时,覆盖的系统动态库可能污染可执行文件 ABI 。
        \item [无法共存] 由于替换了库文件,多个这类驱动显然无法共存。
    \end{labeling}
\end{frame}

\begin{frame}
    \frametitle{API 实现库:奇技淫巧之侵入 Mesa}
    TODO
\end{frame}

\begin{frame}
    \frametitle{API 实现库:ICD (Installable Client Driver) 革命}
    首先定义于 OpenCL 扩展 cl\_khr\_icd ,并被 Vulkan 所继承;类似的机制之后被 NVIDIA 实现于 GL(ES) ,则是 GLVND 。\\
    ICD 机制下,API 实现与接口得到了分离,应用程序链接到 ICD 加载器库,具体驱动程序则在运行时被选择并加载。\\
    ICD 驱动程序的安装过程,则是简单的放置一个 json 配置文件和配置文件中对应名称的动态库文件。
\end{frame}

\begin{frame}
    \frametitle{API 实现库:来自移动端的小小震撼}
    现如今,几乎所有 non-x86 SoC 的渲染/解码子系统和一些新兴厂商的 PCIe GPU ,均采用了某些在移动端更为流行的 GPU IP (Imagination PowerVR , ARM Mali , VeriSilicon Vivante 等)。\\
    非常不幸地,由于这些 GPU IP 现阶段的市场主要在移动端(尤其 Android ),其 API 实现情况大大受移动端生态影响……
\end{frame}

\begin{frame}
    \frametitle{API 实现库:没有桌面 GL 的世界}
    移动端 GPU 驱动的一个通病是:没有桌面版 GL 实现。\\
    然而很不幸的是,现如今的 Linux 桌面可 GLES 化程度仍然不足:\\
    \begin{labeling}{wxWidgets 程序}
        \item [wxWidgets 程序]
    \end{labeling}
\end{frame}

\begin{frame}
    \frametitle{API 实现库:}
\end{frame}

\section{DDX}

\begin{frame}
    \frametitle{DDX:快跑!}
    \begin{alertblock}{TL;DR}
        为了 Wayland 化的将来考虑,在现代 GPU 的新的驱动上请勿使用 DDX 。
    \end{alertblock}
\end{frame}

\begin{frame}
    \frametitle{DDX:没有了会怎么样?}
\end{frame}


\end{document}
