\documentclass{ctexbeamer}
\usetheme{Berlin}

\usepackage{scrextend}

\title{GPU 驱动的 Linux 桌面发行版适配}
\author{Icenowy Zheng}
\institute{PLCT 实验室}
\date{2023-03-17}

\begin{document}

\frame{\titlepage}

\begin{frame}
    \frametitle{内容大纲}
    \begin{itemize}
        \item 自底向上看 GPU 驱动
        \item 开源 GPU 驱动的组成
        \item 厂商 GPU 驱动适配问题
    \end{itemize}
\end{frame}


\part{自底向上看 GPU 驱动}
\frame{\partpage}

\begin{frame}
    \frametitle{内核驱动}
    \begin{block}{“众所周知”}
        只有内核才有直接访问硬件的权限——所谓“用户空间驱动”的存在永远基于内核提供的框架。
    \end{block}
    \begin{itemize}
        \item 显示管线的配置
        \item 显存管理
        \item GPU 任务的提交
        \item 提供额外特性的接口(如硬件编解码等)
    \end{itemize}
\end{frame}

\begin{frame}
    \frametitle{API 实现库}
    \begin{labeling}{通用计算}
        \item [3D 渲染] OpenGL ES/OpenGL/Vulkan
        \item [2D 渲染] OpenVG (及 3D 渲染库)
        \item [资源管理] GBM/EGL/GLX
        \item [硬件解码] VA-API/VDPAU/NV\{ENC,DEC\}
        \item [通用计算] OpenCL/Vulkan/CUDA
    \end{labeling}
\end{frame}

\begin{frame}
    \frametitle{(可选)DDX (Device-Dependent X)}
    \begin{alertblock}{注意}
        DDX 仅在 XFree86 X 服务器下适用。
    \end{alertblock}
    显示相关的 DDX 实现了 X11 中,与硬件相关的部分加速特性。
    \begin{block}{DDX 实现的特性举例}
        \begin{itemize}
            \item DRI 认证/缓冲分配(对于 X 客户端调用图形加速所必须)
            \item Xv 叠加层、XRender 2D 加速等特异性加速
        \end{itemize}
    \end{block}
\end{frame}

\begin{frame}
    \frametitle{(可选)实用工具}
    八仙过海,各显神通。
    \begin{block}{举例}
        \begin{labeling}{状态监测类}
            \item [状态监测类] rocm-smi, nvidia-smi
            \item [性能分析类] rocprof, nvprof
        \end{labeling}
    \end{block}
\end{frame}

\part{开源 GPU 驱动的组成}
\frame{\partpage}

\section{内核接口}

\begin{frame}
    \frametitle{内核接口:FBDEV (Framebuffer Device)}
    \begin{alertblock}{注意:已废弃}
        内核 FBDEV 子系统已废弃,将不会接受新设备驱动支持;DRM 设备仍可提供 FBDEV 兼容设备节点。
    \end{alertblock}
    真·亮机接口。\\
    仅支持 2D 显示一个简单的 Framebuffer ,没有任何现代化的加速功能。\\
    现在仍旧提供的一大原因是 FBCON (Framebuffer Console)。
\end{frame}

\begin{frame}
    \frametitle{内核接口:DRM (Direct Rendering Manager)}
    最早为了 3D 渲染加速创造的接口;\\
    后来随着 KMS (Kernel Mode Setting) 的引入,也开始能控制屏幕显示;\\
    目前开源显示和 GPU 驱动的金标准。\\
    \begin{block}{不同驱动间的接口差异}
        DRM 驱动的 KMS 部分基本是标准化的(也允许在此基础上扩展);但是显存管理、任务管理等均为驱动私有 ioctl 接口,由各个驱动自行定义。
    \end{block}
\end{frame}

\section{API 实现库}

\begin{frame}
    \frametitle{API 实现库:Mesa}
    以一个软件的类 OpenGL\textregistered 实现开局;\\
    后来获得了 DRI 的加成,能够使用图形硬件;\\
    再后来有了基于可编程管线的 Gallium ,能够让不同驱动之间共享大量基础设施;\\
    时至今日,支持的 API 不断扩张,支持了:\\
    \begin{itemize}
        \item GBM/GLX/EGL
        \item OpenGL (ES 和桌面)/Vulkan
        \item OpenCL
        \item VA-API/VDPAU
        \item 以及其他……
    \end{itemize}
\end{frame}

\section{DDX}

\begin{frame}
    \frametitle{DDX:xf86-video-modesetting}
    “那个”通用的 DDX 。\\
    现已成为 X.Org Server 代码库的一部分。\\
    基于通用的 API 实现:\\
    \begin{itemize}
        \item DRM KMS 内核接口:实现显示管理、Vsync
        \item GBM:实现显存管理
        \item OpenGL (ES):实现 2D 加速(含 2D 渲染加速和视频叠加)
    \end{itemize}
    其中基于 OpenGL (ES) 实现 X11 2D 加速的模块被称为 Glamor 。
\end{frame}

\begin{frame}
    \frametitle{DDX:其他}
    其他许多开源驱动同样也都有对应的 DDX:
    \begin{itemize}
        \item xf86-video-intel
        \item xf86-video-ati
        \item xf86-video-amdgpu
        \item xf86-video-nouveau
    \end{itemize}
    特异 DDX 的主要意义是 2D 加速。
\end{frame}

\part{厂商 GPU 驱动适配问题}
\frame{\partpage}

\section{内核驱动}

\begin{frame}
    \frametitle{闭源内核驱动:也是一件精细活}
    \begin{block}{众所周知}
        Linux 内核不维护内核内部 API 的稳定性。\\
        内核会检查 .ko 文件中的版本信息及其暴露的 API,并会拒绝不与当前内核匹配的 .ko 文件。
    \end{block}
    发行闭源驱动的方式:
    \begin{itemize}
        \item 直接丢 .ko 出去——被三体人锁死到对应的内核二进制,需要发行版内核的源码,并与发行版发布内核的过程同步
        \item 将重要文件编译成 .o 发布——由于内核 API 不稳定,这些文件中对内核 API 的调用依然不保证跨内核版本可用
        \item 同上,但精心设计 .o 的导入/导出接口——这种情况能够构造出可以跨内核版本的闭源驱动
    \end{itemize}
\end{frame}

\begin{frame}
    \frametitle{闭源内核驱动:许可证问题}
    \begin{block}{众所周知}
        Linux 内核使用 GPLv2 许可证释出,有授权传递性问题。
    \end{block}
    虽然闭源驱动并未被 Linux 赶尽杀绝,但它们的生存环境依然十分艰难……\\
    \begin{block}{GPL 符号限制:EXPORT\_SYMBOL\_GPL}
        Linux 内核将大量接口以 EXPORT\_SYMBOL\_GPL 这个宏导出,而这样的接口将只能被在元数据中声明 GPL 兼容许可证的模块调用。与之相对的则是 EXPORT\_SYMBOL 宏。
    \end{block}
\end{frame}

\begin{frame}
    \frametitle{闭源内核驱动:接口限制来源}
    Linux 内核中的 GPL 符号限制源于 GPL 中 \textit{derivative work} 的概念,在后文中,我们将被限制为 GPL 的符号称为“仅 GPL 符号”。\\
    仅 GPL 符号一旦被一个模块调用,则该模块会被认为受到 GPL 的授权传递性影响,从而将受到 GPL 的约束 ,因此只有声明了 GPL 兼容许可证的模块才可以调用仅 GPL 符号。\\
    一般来说, GPL 符号限制应用于 Linux 中一些较有特色的子系统(虽然此方面并无明确标准)。
\end{frame}

\begin{frame}
    \frametitle{闭源内核驱动:观赏厂商的表演}
    为了绕过 Linux 内核所做的限制,各种闭源 GPU 驱动厂商采用了一些计策,常见的有如:\\
    \begin{labeling}{计无付之}
        \item [计无付之] 基于内核允许闭源模块调用的接口苦苦编写驱动
        \item [掩耳盗铃] 在内核模块元数据中声明许可证为 GPL ,即便给出的驱动包含闭源部分
        \item [画地为牢] 将闭源部分作为一个单独的模块,编写额外的声明为 GPL 的内核模块作为包装,这一包装模块通过同时引用闭源模块和仅 GPL 符号完成驱动的工作
    \end{labeling}
\end{frame}

\begin{frame}
    \frametitle{闭源内核驱动:绕过手段真的合理吗?}
    \begin{block}{掩耳盗铃}
        当分发这一“声称为 GPL”的内核模块时,如何满足 GPL 对源码可用性的要求呢?
    \end{block}
    \begin{block}{画地为牢}
        曾被某些厂商使用(现已转投下一页所讲方式实现需要 GPL 的功能)。\\
        此方法已失效:自 Linux 5.9 起,如果一个 GPL 兼容的模块(“包装模块”)引用了非 GPL 兼容模块,则前者也将被视为非 GPL 兼容模块,从而无法使用仅 GPL 符号。\\
        Greg Kroah-Hartman 锐评:
        \begin{quote}
            Ah, the proven-to-be-illegal "GPL Condom" defense :)
        \end{quote}
    \end{block}
\end{frame}

\begin{frame}
    \frametitle{内核驱动:开源与闭源的微妙平衡}
    \begin{block}{嵌入式 GPU}
        嵌入式 GPU 的通常做法是将需要保密的逻辑编写为用户空间驱动,同时编写开源(并在芯片 BSP 中附带)的内核驱动配合闭源的用户空间驱动的工作。\\
        ARM Mali GPU 同时在他们的开发者网站上公开提供未适配特定 SoC 的内核驱动以作为参考。
    \end{block}
    \begin{block}{NVIDIA 的 open-gpu-kernel-modules}
        自 Turing (GeForce 系列对应 RTX20/GTX16) 起,NVIDIA 在 GPU 中嵌入了 GSP 处理器,从而能够将需要保密的逻辑移动到 GSP 固件中,而内核驱动本身则不必闭源。
    \end{block}
\end{frame}

\begin{frame}
    \frametitle{内核驱动:警惕无序开发 /dev 的行为}
    另外值得注意的是,很多厂商驱动会在标准的 /dev/fbX 和 /dev/dri/ 之外,创建额外的私有设备节点。\\
    请注意这些私有设备节点的权限,未正确配置可能导致只有 root 用户可以使用 GPU。\\
    编写额外 Udev 规则可以配置这些设备节点的权限,一般来说归属于 root:video 且权限掩码为 660 。\\
    若额外的设备节点接口设计不合理,可能成为无需提权即可利用的安全漏洞。\\
\end{frame}

\section{API 实现库}

\begin{frame}
    \frametitle{API 实现库:李代桃僵}
    部分厂商驱动直接提供了 lib\{GL,EGL,GLESv2,gbm\}.so 用于替换系统原有的 Mesa 库文件;例如 NVIDIA 的 340.xx 旧卡驱动。\\
    这种替换系统库文件的厂商驱动危害如下:\\
    \begin{labeling}{污染二进制文件}
        \item [污染包管理器] 若不通过包管理器安装,则有驱动被覆盖之虞;若通过包管理安装,则需要在打包时引入替换机制。
        \item [污染二进制文件] 由于系统原有的动态库被覆盖,在链接可执行文件时,可能污染可执行文件的 ABI 。
        \item [无法共存] 由于替换了库文件,多个这类驱动显然无法共存。
    \end{labeling}
\end{frame}

\begin{frame}
    \frametitle{API 实现库:奇技淫巧之侵入 Mesa}
    因为历史原因,Mesa 内部也不是一块钢板——它内部有一个 DRI 接口,提供 *\_dri.so 动态文件供 X 服务器实现 AIGLX 。\\
    部分闭源驱动,为了规避替换 Mesa 的库文件的诸多弊端,选择提供一个接入 DRI 接口的库文件。\\
    然而很不幸的是, DRI 接口从未获得妥善的文档;而且随着 Mesa 之外的 DRI 接口使用逐渐消亡,DRI 接口也逐渐转向 Mesa 内部接口,并被破坏性地简化。\\
    也就是说,使用 DRI 接口是一种未定义的操作,且随着发行版对 Mesa 的更新,这一方法很可能于可预见的将来失效。
\end{frame}

\begin{frame}
    \frametitle{API 实现库:ICD (Installable Client Driver) 革命}
    首先由 OpenCL 扩展 cl\_khr\_icd 提出并定义,随后被 Vulkan 所继承;类似的机制之后还被 NVIDIA 实现于 GL(ES) ,即 GLVND 。\\
    ICD 机制下,API 实现与接口得到了分离,应用程序只需链接到 ICD 加载器库,由后者负责在运行时选择并加载具体的驱动程序。\\
    安装一个 ICD 驱动程序是很简捷的:提供驱动的动态库文件,并使用一个 json 配置文件记载这些动态库文件的路径和版本等信息,由此即可被 ICD 加载器库识别。
\end{frame}

\begin{frame}
    \frametitle{API 实现库:来自移动端的小小震撼}
    现如今,几乎所有 non-x86 SoC 的渲染/解码子系统和一些新兴厂商的 PCIe GPU ,均采用了某些在移动端更为流行的 GPU IP (Imagination PowerVR, ARM Mali, VeriSilicon Vivante 等)。\\
    非常不幸地,由于这些 GPU IP 现阶段主要面向移动端(尤其是 Android )市场,其 API 实现情况大大受移动端生态影响……
\end{frame}

\begin{frame}
    \frametitle{API 实现库:没有桌面 GL 的世界}
    \begin{block}{移动端 GPU 驱动的通病}
        没有实现桌面版 GL。
    \end{block}
    此类情况下,通常需要发行版为仅使用 GLES 提供适配。 \\
    部分应用程序(如 KWin)和库(如 Qt)在编译时可以选择使用 GLES ,而部分库(如 wxWidgets)并不具有可用的 GLES 实现。\\
    \begin{block}{GL4ES}
        GL4ES 是一个在 GLES 上部分实现桌面 GL API 的包装库;但是很不幸地,它的 Shader 翻译器过于幼稚,以至于大量应用程序无法运行/需要特别适配……
    \end{block}
\end{frame}

\begin{frame}
    \frametitle{API 实现库:Zink,是救赎还是画饼?}
    Vulkan,作为一种理论上比 GL 更接近底层的图形 API ,可以在其上实现 GL ——事实上 Mesa 已经提供了一个在 Vulkan 基础上实现 GL API 的驱动,也就是 Zink 。\\
    非常不幸的是,移动端 GPU 驱动的 Vulkan 实现通常不好,在桌面 Linux 上表现尤其差……\\
    \begin{block}{某模拟器开发团队锐评}
        \begin{quote}
            It is obvious that only 4 vendors have the expertise and the commitment to make Vulkan drivers work: NVIDIA, AMD, and Mesa, with a special mention for Intel, who recently stepped up their game.
        \end{quote}
    \end{block}
\end{frame}

\begin{frame}
    \frametitle{API 实现库:同样的痛苦,但这次是编解码}
    Android 采用的编解码 API 是 OpenMAX (以下简称 OMX ),很不幸地,这一 API 在 Linux 桌面下应用甚少——Linux 桌面下主要使用的是 VA-API/VDPAU 。\\
    GStreamer 可以提供对 OMX 的支持,但是很不幸地,大多数多媒体应用软件并不支持 OMX 。\\
    更加糟糕的消息是,并没有可用的 API 转换层能够基于 OMX 驱动提供 VA-API/VDPAU……
\end{frame}

\section{DDX}

\begin{frame}
    \frametitle{DDX:快跑!}
    \begin{alertblock}{TL;DR}
        为了 Wayland 化的将来考虑,理论上,新的 GPU 驱动不应当使用 DDX 。
    \end{alertblock}
    DDX 仅是 XFree86 X 服务器的一部分;也就是说,非 X11 的场景下 DDX 起不到丝毫作用。尽管 Wayland 为了兼容 X11 应用程序,提供了 Xwayland 服务器,但后者同样不支持 DDX ,而选择直接调用 Glamor 完成 2D 加速。
\end{frame}

\begin{frame}
    \frametitle{DDX:没有了会怎么样?}
    但是很不幸地,当下 xf86-video-modesetting / Glamor 亦可能遇到各种问题:
    \begin{itemize}
        \item 已在 X 服务器 master 得到修复的问题(尚未发版): modesetting DDX 缺乏双缓冲支持,在性能不足时可能出现撕裂;Glamor 对 GLES 的支持有问题,会导致颜色不正确(蓝色变成黄色)。
        \item Glamor 原理导致的问题: Glamor 大量依赖于 GL 中的 FBO (Frame Buffer Object,帧缓冲区对象),以至于显示缓冲区本身都是用 EGLImage 导入并创建为 FBO 的。这对驱动程序的 FBO 处理提出了很高的要求,在不完善的驱动程序上很可能导致性能问题乃至渲染正确性问题。
    \end{itemize}
\end{frame}

\begin{frame}
    \frametitle{DDX:ABI 版本号}
    对于以闭源形式提供 DDX 模块的驱动来说,还有一个额外的噩耗:\\
    DDX 作为 XFree86 X 服务器的内部细节,并不保证 X 服务器大版本之间的接口稳定性——事实上当下 master 分支的 X 服务器已经具有了和 21.1 分支不同的 ABI 版本号 (26 vs 25)。
\end{frame}

\part{结语与展望}
\frame{\partpage}

\begin{frame}
    \frametitle{驱动适配总结}
    \begin{labeling}{用户空间驱动}
        \item [内核驱动] 两条路:封装核心代码为 .o 文件/全部开源(可以带有闭源固件)
        \item [用户空间驱动] 感恩的心,感谢 ICD 机制,GL 驱动是时候适配 GLVND 了;纯 GLES 驱动仍然是修罗场
        \item [DDX] 不该有,但现阶段很多时候又不得不有的东西
    \end{labeling}
    \hfill \break
    一言蔽之:摸着 NVIDIA 过河。
\end{frame}

\begin{frame}
    \frametitle{展望与许愿}
    能支持开源驱动的话,那是最棒的了!
    不过,如果条件不允许的话:
    \begin{itemize}
        \item 至少开源内核驱动
        \item 至少给图形库适配 GLVND 并适配桌面 GL(或妥善适配 Vulkan )
        \item 至少做好 Glamor 优化
    \end{itemize}
\end{frame}

\begin{frame}
    \frametitle{鸣谢}
\end{frame}

\end{document}
