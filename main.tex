\documentclass{ctexbeamer}
\usetheme{Berlin}

\usepackage{scrextend}

\title{GPU 驱动的发行版适配}
\author{Icenowy Zheng}
\institute{PLCT 实验室}
\date{2023-03-17}

\begin{document}

\frame{\titlepage}

\section{从底向上看 GPU 驱动}

\begin{frame}
    \frametitle{内核驱动}
    \begin{block}{众所周知}
        只有内核才有直接访问硬件的权限——所谓“用户空间驱动”的存在永远基于内核提供的框架。
    \end{block}
    \begin{itemize}
        \item 显示管线的配置
        \item 显存管理
        \item GPU 任务的提交
        \item 提供额外特性的接口(如硬件编解码等)
    \end{itemize}
\end{frame}

\begin{frame}
    \frametitle{API 实现库}
    \begin{labeling}{通用计算}
        \item [3D 渲染] OpenGL ES/OpenGL/Vulkan
        \item [2D 渲染] OpenVG (及 3D 渲染库)
        \item [资源管理] GBM/EGL
        \item [硬件解码] VA-API/VDPAU/NV\{ENC,DEC\}
        \item [通用计算] OpenCL/Vulkan/CUDA
    \end{labeling}
\end{frame}

\begin{frame}
    \frametitle{(可选)DDX (Device-Dependent X)}
    \begin{alertblock}{注意}
        DDX 仅在 XFree86 X 服务器下适用。
    \end{alertblock}
    DDX 实现了 X11 中,与硬件相关的部分加速特性。
    \begin{block}{DDX 实现的特性举例}
        \begin{itemize}
            \item DRI 认证/缓冲分配(对于 X 客户端调用图形加速所必须)
            \item Xv 叠加层、XRender 2D 加速等特异性加速
        \end{itemize}
    \end{block}
\end{frame}

\begin{frame}
    \frametitle{(可选)实用工具}
    八仙过海,各显神通。
    \begin{block}{举例}
        \begin{labeling}{状态监测类}
            \item [状态监测类] rocm-smi, nvidia-smi
            \item [性能分析类] rocprof, nvprof
        \end{labeling}
    \end{block}
\end{frame}

\section{开源 GPU 驱动的组成}

\begin{frame}
    \frametitle{内核接口:FBDEV}
    \begin{alertblock}{注意:已废弃}
        内核 FBDEV 子系统已废弃,将不会接受新设备驱动支持;DRM 设备仍可提供 FBDEV 兼容设备节点。
    \end{alertblock}
    真·亮机接口。\\
    仅支持 2D 显示一个简单的 Framebuffer ,没有任何现代化的加速功能。\\
    现在仍旧提供的一大原因是 FBCON 。
\end{frame}

\begin{frame}
    \frametitle{内核接口:DRM (Direct Rendering Manager)}
    最早为了 3D 渲染加速创造的接口;\\
    后来随着 KMS 的引入,也开始能控制显示;\\
    目前开源显示和 GPU 驱动的金标准。\\
    \begin{block}{不同驱动间的接口差异}
        DRM 驱动的 KMS 部分基本是标准化的(虽然允许扩展);但是显存管理、任务管理等均为驱动私有 ioctl 接口,由各个驱动自行定义。
    \end{block}
\end{frame}

\begin{frame}
    \frametitle{API 实现库:Mesa}
    以一个软件的类 OpenGL\textregistered 实现开局;\\
    后来获得了 DRI 的加成,能够使用图形硬件;\\
    再后来有了 Gallium ,能够让不同驱动之间共享大量基础设施;\\
    时至今日,支持的 API 不断扩张,支持了:\\
    \begin{itemize}
        \item GBM/GLX/EGL
        \item OpenGL (ES 和桌面)/Vulkan
        \item OpenCL
        \item VA-API/VDPAU
        \item 以及其他……
    \end{itemize}
\end{frame}

\begin{frame}
    \frametitle{DDX:xf86-video-modesetting}
    “那个”通用的 DDX 。\\
    现已成为 X.Org Server 代码库的一部分。\\
    基于通用的 API 实现:\\
    \begin{itemize}
        \item DRM KMS 内核接口:实现显示管理、Vsync
        \item GBM:实现显存管理
        \item OpenGL (ES):实现 2D 加速(含 2D 渲染加速和视频叠加)
    \end{itemize}
    其中基于 OpenGL (ES) 实现 X11 2D 加速的模块被称为 Glamor 。
\end{frame}

\begin{frame}
    \frametitle{DDX:其他}
\end{frame}

\section{闭源 GPU 驱动适配问题}

\begin{frame}
    \frametitle{内核驱动:如何闭源?}
\end{frame}

\begin{frame}
    \frametitle{API 实现库:桌面 GL 之殇}
\end{frame}

\end{document}
